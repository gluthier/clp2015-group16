\subsection{Semicolon inference}

This extension allows us to write code without semicolons at the end of the lines. The semicolon becomes an optional element in a statement. It is however still necessary to separate multiple statements on the same line of code.

\begin{lstlisting}
object SemiColonInference {
    def main() : Unit = {
        println(new Tester().test())
    }
}

class Tester {
    def test() : Int = {
		var a : Int
		var b : Int

		a = 1; b = 2

		a =
			a
			+ b
			* 4

		return a
    }
}
\end{lstlisting}

\subsection{Methods as infix operators}

This extesion allow us to call methods not only in prefix form but also as infix operators. It is only allowed to use this form when methods have one argument.

\begin{lstlisting}
object InfixMethods {
    def main() : Unit = {
        println(new Tester().test())
    }
}

class Tester {
    def test() : Int = {
		var myObject : Object
		myObject = new Object().init()

		return myObject infix_method 6
    }
}

class Object {
	var value : Int

	def init() : Object = {
		value = 14
		return this
	}

	def infix_method(a: Int) : Int = {
		return value + a
	}
}
\end{lstlisting}

\subsection{Expressions as statements}

This extension allow to use expressions also when we previously expected a statement.

\begin{lstlisting}

\end{lstlisting}

\subsection{Operators as methods}

This extension allow us to name methods with operators, such as \lstinline$||$, \lstinline$&&$, \lstinline$<$, \lstinline$==$, \lstinline$+$, \lstinline$-$, \lstinline$*$, \lstinline$/$ or \lstinline$!$.

\begin{lstlisting}
object OperatorsAsMethods {
    def main() : Unit = {
        println(new Tester().test())
    }
}

class Tester {
    def test() : Int = {
		var a: ArbitraryObject
		var b: ArbitraryObject

		a = new ArbitraryObject().init()
		b = new ArbitraryObject().init()

		return a.+(b)
    }
}

class ArbitraryObject {
	def init() : ArbitraryObject = {
		return this;
	}

	def +(b: ArbitraryObject): Int = {
		return 100
	}
}
\end{lstlisting}
